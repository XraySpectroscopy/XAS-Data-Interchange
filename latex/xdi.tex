\documentclass{article}
\usepackage{fancybox}           % shadow and oval boxes
\usepackage{amsmath, amssymb}   % AMS mathematics
\usepackage{graphicx}
\usepackage{color}
\usepackage{fancybox}         % shadow and oval boxes
\definecolor{purple}{rgb}{.48, .31, .69}
\usepackage[colorlinks=true, linkcolor=blue, urlcolor=purple]{hyperref}

\setlength{\hoffset}{0cm}
\setlength{\voffset}{0cm}
\setlength{\topmargin}{0cm}
\setlength{\marginparsep}{0.25cm}
\setlength{\marginparwidth}{2.5cm}
\setlength{\textheight}{22cm}

\setlength{\textwidth}{16.5cm}
%\setlength{\topmargin}{-1cm}
\setlength{\oddsidemargin}{0cm}
\setlength{\evensidemargin}{0cm}
\setlength{\headsep}{0.5cm}
\setlength{\parindent}{0.0pt}
\setlength{\parskip}{1.0em}

\newcommand{\sltt}[1]{\texttt{\textsl{#1}}}
\newcommand{\DAQ}{data acquisition}
\newcommand{\DAN}{data analysis}
\newcommand{\xdi}{\textsf{XDI}}
\newcommand{\xdiversion}{1.0}

\newcommand{\mcomm}[2]{\marginpar{\quad\includegraphics[width=1.0cm]{exclamation.jpg}\\%
    See p.~\pageref{#2}\\\textit{#1}}}

\newenvironment{Boxedminipage}%
    {\begin{Sbox}\begin{minipage}}%
    {\end{minipage}\end{Sbox}\shadowbox{\TheSbox}}


\newenvironment{HeaderField}[3]%
{\vspace{1ex}%
    \noindent$\blacklozenge$\quad{\textbf{#1:}}%
    \quad\texttt{[#2]}\quad\dotfill\quad\texttt{\textsl{#3}}%
    \\[0.5ex]%
}{}

\newenvironment{Boxedshadowpage}%
{\begin{Sbox}\begin{minipage}}%
    {\end{minipage}\end{Sbox}\shadowbox{\TheSbox}}
\newenvironment{Coloredbox}[2]%
{\color{#1}\begin{Boxedshadowpage}[h]{#2}\color{black}\vspace{0.3ex}\begin{center}}%
    {\vspace{0.5ex}\end{center}\end{Boxedshadowpage}}

\definecolor{MyGray}{rgb}{0.5,0.5,0.55}
\newenvironment{OpenIssue}%
{\begin{center}\begin{Coloredbox}{MyGray}{0.8\linewidth}%
      \noindent{\textbf{Open Issue:}}\it
    }%
{\end{Coloredbox}\end{center}}
\newenvironment{Todo}%
{\begin{center}\begin{Coloredbox}{MyGray}{0.8\linewidth}%
      \noindent{\textbf{To do:}}\it
    }%
{\end{Coloredbox}\end{center}}




\begin{document}

\begin{center}
  \huge
  The XAS Data Interchange Format\\
  Draft Specification, version \xdiversion\\[4ex]
  \Large
  XDI Working Group\\[2ex]
  \normalsize
  Bruce Ravel (NIST) bravel\char64 bnl.gov\\
  ...and...
\end{center}

\tableofcontents

\section*{Typography notes}
\label{typography}

In this draft, an exclamation symbol in the margin, shown to the
right, indicates that an open issue related to that text is discussed
in Sec.~\ref{sec:future}.\mcomm{}{sec:future}

\newpage

\section{Introduction}
\label{sec:introduction}
% FIXME: need an introductory blurb

\subsection{Purpose}
\label{sec:purpose}

This document describes the XAS Data Interchange Format ({\xdi}),
version \xdiversion, a simple file format for a single X-ray
Absorption Spectroscopy (XAS) measurement.  We are defining this
format to accomplish the following goals:

\begin{enumerate}
\item Establish a common language for transferring data between XAS
  beamlines, XAS experimenters, data analysis packages, web
  applications, and anything else that needs to process XAS data.
\item Increase the relevance and longevity of experimental data by
  reducing the amount of \textit{data archeology}\footnote{The Farrel
    Lytle database
    (\href{http://ixs.csrri.iit.edu/database/data/index.html}
    {\texttt{http://ixs.csrri.iit.edu/database/data/index.html}}) is a
    particularly trenchant example of data archeology.} future
  interpretations of that data will require.
\item Enhance the user experience by promoting interoperability among
  data acquisition systems and data analysis packages.
\item Provide a mechanism for extracting and preserving a single
  XAS-like data set from a related experiment (for example, A DAFS
  measurement) or from a complex data structure (for example, a
  database or a hierarchical data file used to store a multispectral
  data set).
\end{enumerate}

This format is intended to encode single-scan data files with
metadata.  It is not intended to encode relationships between many XAS
measurements or between an XAS measurement and other parts of a
multispectral experiment.

In order to fulfill these goals, {\xdi} files provide a flexible,
consistent representation of information common to all XAS
experiments.  This format is simpler than a format based on XML, HDF,
or a database; it yields self-documenting files; and it is easy for
both humans and computers to read.  Its structure is inspired by that
of Internet electronic mail\footnote{See \textit{RFC822: Standard for
    ARPA Internet Text Messages},
  \href{http://www.w3.org/Protocols/rfc822/}
  {\texttt{http://www.w3.org/Protocols/rfc822/}}}, a plain-text data
format which has proven to be robust, extensible, and enduring.  Due
to these advantages, and because of our intention to develop free
software tools and libraries that support {\xdi}, we hope that this
file format described in this specification will see wide adoption in
the XAS community.


\subsection{Scope}
\label{sec:scope}

We do not intend this specification to dictate the file formats used
by data acquisition systems during XAS experiments, although this is
certainly a suitable format for that purpose.  Any attempt to do
so would be unreasonable due to the number of different data
acquisition systems currently deployed at synchrotrons around the
world, the variety of experiments performed at these installations,
and the continuing development of new experimental techniques.
Instead, this specification addresses the representation of a single
scan of XAS data after an experiment has been completed.

A beamline which adopts this specification shall either use this
format as its native file format or shall provide their users with
tools that convert between their native file formats and {\xdi}.  In
short, they will send their users home with their XAS data stored in
this format.  We intend to encourage this practice by developing tools
for reading, editing, writing, and validating {\xdi} files.  Beamlines
may choose to modify their data acquisition systems to write data
using this format in situations where that would be appropriate.  We
plan to assist in this effort by developing libraries for popular
programming languages which can read, manipulate, and write {\xdi}
files.

With their experiment data stored in {\xdi} files, users may choose
data analysis packages which are capable of reading this format.  It
is our hope that, as this specification gains wider adoption, users
will ultimately be freed from the responsibility of understanding file
formats.  With this aim in mind, we shall assist software developers
in supporting {\xdi} files.


\section{The Contents of the XDI File}
\label{sec:contents_ixsif}

{\xdi} files contain two sections, a header with information about one
scan of an XAS experiment and the data collected during that scan.
The header consists of versioning information, a series of fields that
contain a single pieces of information, an area for users to store
comments about the experiment, and a sequence of labels for the
columns of data.  The data section contains these columns, with each
row corresponding to one point of the scan.

Although the header has been designed to contain arbitrary
information, the meanings of several fields are explicitly defined.
These fields, described below, contain the most common information
about XAS experiments.  We hope that users will benefit from their
existence when using data analysis packages that support {\xdi}
files.  However, none of the defined fields are required to be
present.  For example, some of these fields may not be appropriate for
certain experiments and should be omitted in that case.

Some examples of header information follow.  A complete list of
defined headers along with their specifications is found in
Sec.~\ref{sec:ixsif_fields_standard}.

\begin{description}
\item[Beamline:] the location where the experiment was performed.
\item[Crystal:] information about the monochromator used in the
  experiment.
\item[Edge Energy:] edge energy value defined by the data acquisition
  software.
\item[Focusing:] comments about the focusing optics used in the
  experiment.
\item[Harmonic:] undulator harmonic used in the experiment.
\item[Ring Current:] current of the synchrotron's storage ring.
\item[Ring Energy:] energy of the synchrotron's storage ring.
\item[Source:] the type of x-ray source used in the experiment.
\item[Timestamps:] start and end times of this scan.
\item[Mu expressions:] math expressions for calculating experimental
  spectra from the data columns.
\end{description}


\section{Definintion of the IXS Interchange Format}
\label{sec:def_ixsif}

This section of the {\xdi} specification formally describes the
structure of {\xdi} files.

\subsection{Requirements}
\label{sec:def_requirements}

The key words ``\textbf{must}'', ``\textbf{must not}'',
``\textbf{required}'', ``\textbf{shall}'', ``\textbf{shall not}'',
``\textbf{should}'', ``\textbf{should not}'',
``\textbf{recommended}'', ``\textbf{may}'', and ``\textbf{optional}''
in this document are to be interpreted as described in RFC
2119.\footnote{\textit{Key words for use in RFCs to Indicate
    Requirement Levels} \href{http://www.ietf.org/rfc/rfc2119.txt}
  {\texttt{http://www.ietf.org/rfc/rfc2119.txt}}.}

An {\xdi} implementation is not compliant if it fails to satisfy
one or more of the \textbf{must} or \textbf{required} level
requirements presented in this specification.

\subsection{Notational Conventions}
\label{sec:def_notation}

All of the representations defined in this document are described both
in prose and using an augmented Backus-Naur Form (BNF).  The syntax
used in these grammars is defined in RFC
5234\footnote{\textit{Augmented BNF for Syntax Specifications}
  \href{http://tools.ietf.org/html/rfc5234}
  {\texttt{http://tools.ietf.org/html/rfc5234}}}.
% section 2.1 of the Internet
% Engineering Task Force (IETF)
% \href{http://www.ietf.org/rfc/rfc2616.txt}{Request for Comments (RFC)
%   2616}, ``The Hypertext Transfer Protocol''. 
Software developers who wish to implement support for {\xdi} files
themselves will need to familiarize themselves with this notation to
understand this specification.

The basic rules used throughout this section to define parsing
constructs are presented in the appendix in~\ref{apdx:gram_core} and
\ref{apdx:gram_basic} as part of the complete grammar. All parsing
rules that consist of a sequence of multi-character tokens
\textbf{must} be delimited by white space unless the tokens of the
sequence may be unambiguously identified.

\subsection{Text Encoding}
\label{sec:def_encoding}

The header and data sections of an {\xdi} file are comprised of
structured US-ASCII\footnote{ASCII table:
  \href{http://en.wikipedia.org/wiki/ASCII}
  {\texttt{http://en.wikipedia.org/wiki/ASCII}}} text.  Header field values
that are ``free-form'' or ``text'' \textbf{may} contain UTF-8 encoded
Unicode text, although Unicode support in applications that use {\xdi}
files is OPTIONAL.  The US-ASCII coded character set is formally by
ANSI X3.4-186.  The Universal Character Set (Unicode) is defined by
ISO/IEC 10646.  The UTF-8 translation format is defined by IETF RFC
3629.\mcomm{Unicode}{anchor:unicode}

\subsection{Header Section}
\label{sec:def_hdr}

The header section of an {\xdi} file appears at the beginning of the
file and is comprised of structured text.  Every line of the header
\textbf{must} begin with a comment character and \textbf{must} end
with an end-of-line sequence, both of which are defined below.  There
are no multi-line headers.  Lines may be of any length.  Support for
the Posix, Apple, or Microsoft end-of-line conventions is provided to
increase cross-platform portability.

\begin{verbatim}
        COMM           = "#" | ";"
        EOL            = CR  | LF |  CRLF
\end{verbatim}

Header lines are subdivided into four \textbf{optional} subsections
--- versioning information, header fields, user comments, and column
labels --- with two separators that are \textbf{required} when when
header sections are present.  These are organized in the following
sequence:

\begin{enumerate}
\item The \textbf{optional} first line of the file is the version
  line.
\item This is followed by zero or more header fields, which can be
  \textit{defined} headers or \textit{extension} headers.  These two
  header types are explained in Sec.~\ref{sec:def_hdr_fields}.
\item The header lines are separated from the user comments by the
  \texttt{FIELD-END} rule, which is a comment character followed one
  or more slashes (\texttt{/}) followed by an end-of-line.
\item The comment section is for user-supplied, free-format text.
  Each line begins with a comment character and ends with an
  end-of-line.
\item The comment section ends with the \texttt{HEADER-END} rule,
  which is a line of dashes which starts with a comment character and
  ends with an end-of-line.
\item The last line before the data is a line of column labels which
  identify the columns of data.  There should be as many labels as
  there are columns.  The label line begins with a comment character
  and ends with an end-of-line.
\end{enumerate}

All header sections are \textbf{optional}.  In a file that does not
follow the {\xdi} specification but which contains an obvious header
(obvious in the sense that lines begin with a comment character and
end with an end-of-line), the obvious header lines \textbf{should} be
interpreted as user comments.

The \textbf{optional} status of the headers is to accommodate data
files which contain obvious header lines but which are not compliant
with this specification.  In that case, all header lines are
interpretted as user comments.  Of course, in that case few of the
advantages of the {\xdi} format are realized.

The separator lines (\texttt{FIELD-END} and \texttt{HEADER-END}) serve
specific, syntactic purposes in the {\xdi} grammar.  The line of
dashes is a common visual cue denoting the end of the headers and
beginning of the data.  The \texttt{FIELD-END} serves to separate and
distinguish field lines from freely-formatted user comments, which may
resemble a header fields or other gramatical constructs.  Similarly,
the \texttt{HEADER-END} serves to distinguish column labels from user
comments, which are otherwise grammatically identical elements of the
data file.

\begin{verbatim}
        FIELD-END   = "#"  1*"/"  EOL
        HEADER-END  = "#"  2*"-"  EOL
\end{verbatim}


\subsubsection{Version Information}
\label{sec:def_hdr_version}

The first line of the {\xdi} header contains the {\xdi} version to
which the file conforms.  {\xdi} represents versions of the file
format with a \texttt{<major>.<minor>} numbering scheme.  The
\texttt{<minor>} version is incremented when changes are made to the
format that do not affect compatibility with previous versions, as
when new defined header fields are defined.  (A parser compliant with
an earlier minor version would treat the newly defined header as an
extension field.  Propagated to an output file as an extension field,
this field would then be interpretted correctly by a more recent
parser.)  The \texttt{<major>} version is incremented when other
changes are made to the format, as when the definition of the contents
of a defined header field is altered.

A series of optional version entries, separated by white space, may
follow the {\xdi} version.  These version entries exist to allow
various programs to annotate the file as it proceeds through the
collection and analysis process. Such annotation is \textbf{optional}
although version information \textbf{must} be included in this
sequence by software that create {\xdi} files containing extension
fields (see section~\ref{sec:ixsif_fields_extension}).  The order of
the optional version entries is undefined but \textbf{should} be
preserved to accurately represent the sequence in which applications
have manipulated the file.

\begin{verbatim}
        XDI-VERSION     = "XDI/"  1*DIGIT  ". " 1*DIGIT
        APPLICATIONS    = VCHAR
        VERSION         = "#"  XDI-VERSION  *APPLICATIONS  EOL
\end{verbatim}

Note that the {\xdi} major and minor version numbers \textbf{must} be
treated as integers that \textbf{may} contain more than a single
digit.  ``XDI/1.12'' is a higher (more recent) version than
``XDI/1.2''.

This specification does not impose a restriction on how applications
identify and version themselves.  However, a single application
\textbf{must} identify and version itself using a single text sequence
without white space.  Some acceptible examples follow.  The first
example shows an application which uses the same format as the XDI
version rule; the second shows the names of the data acquisition and
data processing programs are specified by name but without version
numbers; the third shows some arbitrary method of versioning an
application.

\begin{verbatim}
     # XDI/1.0 Datacollectatron/7.75

     # XDI/1.0 XDAC Athena

     # XDI/1.0 XAS!Collect-3000
\end{verbatim}


\subsubsection{Header Fields}
\label{sec:def_hdr_fields}

The lines immediately following the version line of the header contain
the fields of the header.  These fields are arranged in a manner
similar to the the header of an Internet electronic mail message,
although {\xdi} fields may not span multiple lines.  Each field
consists of a case-insensitive name, a separating colon, and an
associated value.  When multiple occurrences of the same field are
present the value of the last occurrence \textbf{must} be used as the
value for the field.

Although the values of some fields have a required structure, all
values are assumed to be free-form text in the following rules.  Rules
for each of the defined fields are defined in
section~\ref{sec:ixsif_fields_standard} and the complete definition of
the \texttt{FIELDS} rule may be found in
section~\ref{apdx:gram_hdr_fields}.  Although some defined fields take
more specificly specified contant, the generic definition of a field
looks like this:

\begin{verbatim}
        PROPERWORD     = ALPHA  *(ALPHA  |  DIGIT  |  "_")
        WORD           = *(ALPHA  |  DIGIT  |  "_")
        FIELD-NAME     = PROPERWORD
        FIELD-VALUE    = *WORD
        FIELD-LINE     = "#"  FIELD-NAME  ": "  FIELD-VALUE EOL
        FIELDS         = *FIELD-LINE  FIELD-END
\end{verbatim}

The header fields subsection is ended with a \texttt{FIELD-END} line.
Note that because no fields are required to be present, this
subsection may contain no lines. The dividing line \textbf{must} be
present if any header lines are present but \textbf{may} be absent if
no header lines are present.  Also note that because
\texttt{FIELD-VALUE} matches zero or more \texttt{WORD} terminals, it
is not required to contain any text.

\subsubsection{User Comments}
\label{sec:def_hdr_comments}

Following the dividing line at the end of the header fields subsection
is the area of the header that contains user comments. Please note
that this area is reserved for comments supplied by the experimenters
and \textbf{must not} be used by software as a place to store other
information.  Refer to section~\ref{sec:ixsif_fields_extension} for
information about using extension fields for this purpose.

\begin{verbatim}
        COMMENT-LINE   = "#"  *VCHAR  EOL
        COMMENTS       = *COMMENT-LINE  HEADER-END
\end{verbatim}

As with the header fields, this section may contain no lines of
commentary or lines that contain no comment text but \textbf{must} end
with a dividing line.  When extracting the comment subsection from an
{\xdi} file, software \textbf{should} remove a single leading space
and any trailing white space from each comment line but \textbf{must
  not} further alter the line's contents.

Applications \textbf{must} preserve all comment fields.

\subsubsection{Column Labels}
\label{sec:def_hdr_labels}

The final line of the {\xdi} header contains the labels for each
column of data in the data section of the file, separated by
white space.  There \textbf{must} be one label present for each column
of data present in the data section.

\begin{verbatim}
        LABEL          = *WORD
        LABELS         = "#"  1*LABEL  EOL
\end{verbatim}

The number of column labels \textbf{must} equal the number of columns
of data in the data section.

\subsection{Data Section}
\label{sec:def_data}

The data section of the file contains white space delimited columns of
floating-point numbers.  If the abscissa is not explicitly identified
using the \texttt{ABSCISSA} header, then the first column of this section
\textbf{must} contain the abscissa.  The remaining columns
\textbf{must} correspond to experimental values at that abscissa.  If
the abscissa is not the photon energy, then the \texttt{ABSCISSA}
\textbf{must} define a math expression for converting the abscissa
column to energy.\mcomm{Floats}{anchor:floats}

\begin{verbatim}
        DATA-LINE      = *FLOAT EOL
        DATA           = *DATA-LINE
\end{verbatim}

Note that blank lines in this section \textbf{must} be discarded; the
number of columns \textbf{must} be the same for all lines that contain
data; and any measurements of times present \textbf{must} be
represented as floating point numbers.


\section{XDI Fields}
\label{sec:ixsif_fields}

\subsection{Defined Fields}
\label{sec:ixsif_fields_standard}

When present, the following header fields \textbf{must} comply with
their associated parsing rules.  Any fields which fail to do so
\textbf{must} be ignored by preprocessing and analysis software.  The
text in brackets to the right of the token provides a quick overview
of the expected format, and any text following the line of dots is an
example of a valid value.  The grammar rule for the header follows.

\begin{HeaderField}{Abscissa}{<math expression>}{\$1}
This field identifies the column containing the abscissa of the data
contained in the file.  The math expression can be used to specify how
to convert the specified column to energy.  For instance, if data are
recorded as a function of encoder value, then the math expression
might be something like:
\begin{verbatim}
     12398.61 / (2 * dspacing) / sin( $1 / (57.29577951 * stpdeg) )
\end{verbatim}
where 12398.61 is $\hbar c$ in eV$\cdot${\AA} units, 57.29577951 is
the constant for converting between radians and degrees, and
\texttt{dspacing} and \texttt{stpdeg} would be replaced by the
d-spacing and number of steps per degree for that
monochromator.\mcomm{\texttt{MATH} rule}{anchor:math}
\begin{verbatim}
        ABSCISSA       = "#"  "Abscissa"  ": "  MATH EOL
\end{verbatim}
\end{HeaderField}

\begin{HeaderField}{Beamline}{<text>}{APS 10ID}
The location where the experiment was performed.
\begin{verbatim}
        BEAMLINE       = "#"  "Beamline"  ": "  *WORD EOL
\end{verbatim}
\end{HeaderField}

\begin{HeaderField}{Collimation}{<text>}{fixed-angle, Rh-coated torroid}
A brief description of the collimating mirrors used at the beamline.
\begin{verbatim}
        COLLIMATION    = "#" "Collimation"  ": "  *WORD  EOL
\end{verbatim}
\end{HeaderField}

\begin{HeaderField}{Crystal}{<material> <reflection>}{Si 111}
  Information about the crystals of the monochromator used in the
  experiment.\mcomm{Monochro-\\mators}{anchor:mono}
\begin{verbatim}
        MATERIAL       = (  "Si" / "Ge" / "Diamond" / "YB66" / "InSb" 
                          / "Beryl" / "Multilayer")
        REFLECTION     = 3DIGIT
        CRYSTAL        = "#" "Crystal"  ": " MATERIAL REFLECTION EOL
\end{verbatim}
\end{HeaderField}

\begin{HeaderField}{D\_spacing}{<float>}{3.13555}
The interplaner spacing of the monochromator's crystals, in {\AA}ngstroms.
\begin{verbatim}
        DSPACING       = "#"  "D_spacing"  ": "  FLOAT  EOL
\end{verbatim}
\end{HeaderField}

\begin{HeaderField}{Edge\_energy}{<float>}{5465}
  The energy reference of the scan -- often the zero-valent edge
  energy of the absoring atom -- as defined in the data acquisition
  software, in electron volts.\mcomm{Energy units}{anchor:units}
\begin{verbatim}
        EDGEENERGY     = "#"  "Edge_energy"  ": "  FLOAT  EOL
\end{verbatim}
\end{HeaderField}

\begin{HeaderField}{End\_time}{<timestamp>}{2003-04-01T13:01:02}
  The date and time that this scan ended, using the timestamp format
  specified in ISO 8601\footnote{ISO 8601, \textit{
      Representation of dates and times} is explained succintly at\\
    \href{http://www.iso.org/iso/support/faqs/faqs_widely_used_standards/widely_used_standards_other/date_and_time_format.htm}{http://www.iso.org/iso/support/faqs/faqs\_widely\_used\_standards/widely\_used\_standards\_other/date\_and\_time\_format.htm}}.
  The example above represents one minute and two seconds after one
  o'clock in the afternoon of April 1$^{\mathrm{st}}$,
  2003.\mcomm{Time zones}{anchor:timezones}
\begin{verbatim}
        DATETIME  = 4DIGIT  "-"  2DIGIT  "-"  2DIGIT "T" 2DIGIT  ":"  2DIGIT  ":"  2DIGIT
        ENDTIME   = "#"  "End-time"  ": "  DATETIME  EOL
\end{verbatim}
\end{HeaderField}

\begin{HeaderField}{Focusing}{<text>}{sagitally bent second crystal}
A brief description of the focusing optics used in the experiment.
\begin{verbatim}
        FOCUSING       = "#"  "Focusing"  ": "  *WORD  EOL
\end{verbatim}
\end{HeaderField}

\begin{HeaderField}{Harmonic\_rejection}{<text>}{50\% detuning of 2nd xtal}
A brief description of the harmonic rejection strategy used at the beamline.
\begin{verbatim}
        HARMONIC-REJECTION  = "#" "Harmonic_rejection"  ": "  *VCHAR  EOL
\end{verbatim}
\end{HeaderField}

\begin{HeaderField}{Mu\_fluorescence}{<math expression>}{\$4/\$2}
  The math expression for calculating the $\mu(E)$ of fluorescence
  from this file's data section.\mcomm{\texttt{MATH} rule}{anchor:math}
\begin{verbatim}
        MUFLUOR        = "#"  "Mu_fluorescence"  ": "  MATH  EOL
\end{verbatim}
\end{HeaderField}

\begin{HeaderField}{Mu\_reference}{<math expression>}{ln(\$3/\$5)}
  The math expression for calculating the $\mu(E)$ of the reference
  channel from this file's data section.
\begin{verbatim}
        MUREF          = "#"  "Mu_reference"  ": "  MATH  EOL
\end{verbatim}
\end{HeaderField}

\begin{HeaderField}{Mu\_transmission}{<math expression>}{ln(\$2/\$3)}
  The math expression for calculating the $\mu(E)$ of transmission
  from this file's data section.
\begin{verbatim}
        MUTRANS        = "#"  "Mu_transmission"  ": "  MATH  EOL
\end{verbatim}
\end{HeaderField}

\begin{HeaderField}{Ring\_current}{<float>}{101.2}
The current of the synchrotron's storage ring, in milliamperes.
\mcomm{Ring current}{anchor:current}
\begin{verbatim}
        RINGCURRENT    = "#"  "Ring_current"  ": "  FLOAT  EOL
\end{verbatim}
\end{HeaderField}

\begin{HeaderField}{Ring\_energy}{<float>}{7.01}
  The energy of the synchrotron's storage ring, in giga-electron volts (GeV).
\begin{verbatim}
        RINGENERGY     = "#"  "Ring_energy"  ": "  FLOAT  EOL
\end{verbatim}
\end{HeaderField}

\begin{HeaderField}{Start\_time}{<timestamp>}{2003-04-01T13:01:02}
  The date and time that this scan started, using the timestamp format
  specified in ISO 8601\footnote{ISO 8601, \textit{
      Representation of dates and times} is explained succintly at\\
    \href{http://www.iso.org/iso/support/faqs/faqs_widely_used_standards/widely_used_standards_other/date_and_time_format.htm}{http://www.iso.org/iso/support/faqs/faqs\_widely\_used\_standards/widely\_used\_standards\_other/date\_and\_time\_format.htm}}.
  The example above represents one minute and two seconds after one
  o'clock in the afternoon of April 1$^{\mathrm{st}}$, 2003.\mcomm{Time zones}{anchor:timezones}
\begin{verbatim}
        DATETIME  = 4DIGIT  "-"  2DIGIT  "-"  2DIGIT "T" 2DIGIT  ":"  2DIGIT  ":"  2DIGIT
        STARTTIME = "#"  "End-time"  ": "  DATETIME  EOL
\end{verbatim}
\end{HeaderField}

\begin{HeaderField}{Source}{<text>}{APS undulator A}
  The type of x-ray source used in the experiment.
\begin{verbatim}
        SOURCE         = "#"  "Source"  ": "  *WORD  EOL
\end{verbatim}
\end{HeaderField}

% \begin{HeaderField}{Step\_offset}{<float>}{0.0}
% The photon energy is calculated from the abscissa using the the value of this
% field in the equation $E(a) = a*{\rm scale} + {\rm offset}$.  If the
% \texttt{Step-scale} field is not present, the default value of $1$ is used.
% \begin{verbatim}
%         STEPOFFSET     = "#"  "Step_offset"  ": "  FLOAT  EOL
% \end{verbatim}
% \end{HeaderField}

% \begin{HeaderField}{Step\_scale}{<float>}{1.0}
% The photon energy is calculated from the abscissa using the the value of this
% field in the equation $E(a) = a*{\rm scale} + {\rm offset}$.  If the
% \texttt{Step-offset} field is not present, the default value of $0$ is used.
% \begin{verbatim}
%         STEPSCALE      = "#"  "Step_scale"  ": "  FLOAT  EOL
% \end{verbatim}
% \end{HeaderField}

% \begin{OpenIssue}
%   The step-* fields are probably not the right way to go.  They are
%   intended to help translate from encoder reading to energy, however a
%   different formula is required to translate from angle to energy.
%   Much better would be to define a field which takes an
%   \texttt{MATH} for translating a column to energy.
% \end{OpenIssue}


\begin{HeaderField}{Undulator\_harmonic}{<int>}{3}
The undulator harmonic used in the experiment.\mcomm{Sources}{anchor:sources}
\begin{verbatim}
        HARMONIC-VALUE = 1*2DIGIT
        HARMONIC       = "#"  "Undulator_harmonic"  ": "  HARMONIC-VALUE  EOL
\end{verbatim}
\end{HeaderField}

\subsection{Extension Fields}
\label{sec:ixsif_fields_extension}

Extension fields are fields present in the header of an {\xdi} file
that are not defined in that file's version of {\xdi}.  Such fields
are interpreted as having values of free-form text.  Any field not
defined in section~\ref{sec:ixsif_fields_standard} \textbf{must} be
considered an extension field, providing backwards compatibility
between different minor versions of this specification.

\begin{verbatim}
        EXT-FIELD-NAME = PROPERWORD  *("-"  PROPERWORD)
        EXT-FIELD      = "#"  EXT-FIELD-NAME  ": "  *VCHAR EOL
\end{verbatim}

Data acquisition systems and data analysis packages may embed
additional information in an {\xdi} file by adding extension fields to
the header.  Extension fields created by applications \textbf{should}
begin with a form of the application name used in the version line,
followed by a hyphen (in appendix~\ref{apdx:example} examples such as
\texttt{MX-SSRS} are shown in the example data file, where MX is the
name of the data acquisition software at that beamline).  This
requirement prevents field name collisions between different
applications and between applications and future versions of this
specification.

Applications that read {\xdi} files \textbf{may} attempt to parse the
values of extension fields to extract the additional information about
the scan.  They \textbf{should} propagate these fields into output
files they create, but \textbf{must} propagate the associated version
information if they do so.

When multiple occurrences of the same field are present the value of
the all occurrences \textbf{must} be preserved.  In this way,
extension fields are interpreted differently from defined headers.

\subsection{Grammar for the Header Fields}
\label{sec:ixsif_fields_grammar}

Having defined the rules of the defined header fields, it is now
possible to create a complete version of the \texttt{FIELDS} rule that
was provisionally defined in section~\ref{sec:def_hdr_fields}.  The
complete {\xdi} grammar is found in appendix~\ref{apdx:gram}.
Section~\ref{apdx:gram_hdr_fields} shows the complete definition of
the header fields.

\section{Future prospects}
\label{sec:future}

There are several limitations to the current grammar.  These are
candidates for future development and are presented in no special
order.

\begin{description}
\item[Sample:] Headers for describing the sample and its preparation
  are not provided in the version.
\item[White space:] Use of white space is ambiguous in most rules.
\item[Abscissa:] The \texttt{MATH} rule should include a way of
  specifying a function converting wavelength to energy, something
  like \texttt{k2e(\$1)}.  Also need to recognize \texttt{sin()}, as
  shown in the explanation of the \texttt{ABSCISS}
  field.\label{anchor:math}
\item[Math rule:] The math expression language for providing hints for
  converting columns in useful spectra needs a more complete
  definition.  This should be flexible enough to provide hints for
  doing dead-time corrections.
\item[Energy units:] Other energy units -- eV, keV, {\AA}ngstroms,
  degrees, encoder counts.  Do we need a defined header that defines
  abscissa units?\label{anchor:units}
\item[Other spectroscopies:] {\xdi} would be quite satisfactory for
  single scan files of several other types of spectroscopy, including
  XES, NIXS, DAFS, and ReffleXAFS.  Supporting these types would
  require definition of new defined headers.  Headers similar to
  \texttt{Mu\_tranmission} (i.e.\ headers which take \texttt{MATH}
  rules) could simultaneously identify the spectroscopy type and
  provide the appropriate hint for constructing a spectrum from the
  columns.
\item[Theory:] {\xdi} would be suitable for single-spectrum output
  files from \textsc{feff} and other theory programs.  Metadata for
  these theories is probably best supported by extension fields and
  identifying the theory code can be done in the version line.
  However, a defined header identifying the theory program and hinting
  fields taking \texttt{MATH} rules might be appropriate.
\item[Monochromators:] Should crystal materials be explicitly listed?
  Should it be free-form instead?  What about asymmetric crystals?
  4-bounce monos?  Polychromators?\label{anchor:mono}
\item[Internationalization:] The current draft of {\xdi} is
  English-centric because the initial developers are native English
  speakers.  Internationalization could be implemented in the grammar.
  That is, a line like \texttt{\# Beamline:\ APS 10ID} could be
  localized in Spanish as \texttt{\# Linea:\ APS 10ID}.
  Internationalization of defined headers would require a complete
  (and competent) translation and definition of defined headers using
  alternation, for example \texttt{BEAMLINE\_TAG = ("Beamline" / "Linea")}.

  Note that neither comment lines nor column labels are defined in a
  localized way in the current grammar (i.e.\ they can be in any
  language).  Also note that an octet \textit{is} defined in the
  grammar, so extending to western languages should be trivial.
\item[Unicode:] The current grammar does not explicitly consider
  unicode.  This is an essential future development so that comments,
  column labels, and extension fields are handled correctly anywhere
  in the world.  Internationalization of defined headers will require
  proper handling of unicode.\label{anchor:unicode}
\item[Time zones:] Do we need a header defining time zone for complete
  interpretation of timestamps? Or is the source header sufficient for
  determining location of the measurement?\label{anchor:timezones}
\item[Floating point numbers:] Does the type for the data need to be
  defined, i.e.\ single or double precision or
  otherwise?\label{anchor:floats}
\item[Sources:] The current draft contains defined headers specific to
  undulator sources.  Appropriate headers need to be specified for
  other sources, e.g.\ gap width of a variable-gap wiggler.  Headers
  for unconventional sources (i.e.\ plasma source, laboratory source)
  should be defined.\label{anchor:sources}
\item[Ring current:] In non-top-off mode, the ring current will change
  throughout the scan.  What value should be recorded in the file?
  The current at the beginning of the scan?  At the end?  Something
  else (say, the average throughout the scan)?\label{anchor:current}
\item[Case sensitivity:] RFC 5234 specifies case insensitivity for all
  rules.  This specification uses capitalized words for defined
  headers.  This is permissable by RFC 5234, so long as they are
  specified as explicit strings.  Should defined fields be
  case-insentive?
\item[Target programming languages:] A minimal set of languages for
  which {\xdi} support \textbf{should} be provided to assist in its
  wide-spread adoption includes: C, C++, Fortran, Java, Python, Perl,
  LabView, IDL.  But all language are welcome.
\end{description}

\newpage
\appendix

\section{Example XDI File}
\label{apdx:example}

Here is an example of a file conforming to this specification and
providing substantial metadata.  This was edited by hand from a real
data file measured at beamline ID10 at the APS in 2005.  The lines
beginning \texttt{MX-} are extension fields denoting parameters of the
MX data acquisition system in use at the beamline.

\begin{center}
\begin{Boxedminipage}[h]{0.7\linewidth}
\begin{verbatim}
   # XDI/1.0 MX/2.0
   # Beamline: APS 10ID
   # Source: undulator a
   # Ring_energy: 7.00
   # Undulator_harmonic: 3
   # Crystal: Si 111
   # Collimation: none
   # Focusing: none
   # Harmonic_rejection: flat Rh-coated mirror
   # Start_time 2005-03-08T20:08:57
   # Edge_energy: 7112.00
   # Abscissa: $1
   # Mu_transmission: ln($2/$3)
   # Mu_fluorescence: $4/$2
   # Mu_reference: ln($3/$5)
   # MX-Num-regions: 1
   # MX-SRB: 6900
   # MX-SRSS: 0.5
   # MX-SPP: 0.1
   # MX-Settling-time: 0
   # MX-Offsets: 11408.00 11328.00 13200.00 10774.00
   # MX-Gains: 8.00 7.00 7.00 9.00
   #///
   # Fe K-edge, Lepidocrocite powder on kapton tape, RT
   # 4 layers of tape
   # exafs, 20 invang
   #---
   # energy     mcs3     mcs4     mcs6     mcs5
    6899.9609  48120 19430 2250 54540
    6900.1421  48390 19540 2260 54860
    6900.5449  48520 19610 2250 55110
    6900.9678  48930 19780 2280 55650
    6901.3806  48460 19590 2250 55110
        (....etc....)
\end{verbatim}
\end{Boxedminipage}
\end{center}


\newpage

\section{Grammar of the XDI}
\label{apdx:gram}

\subsection{Start rule}
\label{sec:startrule}

\begin{verbatim}
     XDI            = [VERSION] [FIELDS] [COMMENTS] [LABELS] 1DATA
\end{verbatim}

\subsection{Core rules}
\label{apdx:gram_core}
\begin{verbatim}
     OCTET          = %x00-FF             ; 8 bits of data
     UPALPHA        = %x41-5A             ; upper case letters A - Z
     LOALPHA        = %x61-7A             ; lower case letters a - z
     CHAR           = %x01-7F             ; any 7-bit US-ASCII character, excluding NUL
     VCHAR          = %x21-7E             ; visible (printing) characters, 7-bit (US-ASCII)
     ALPHA          = UPALPHA / LOALPHA   ; US-ASCII letters
     DIGIT          = %x30-39             ; digits 0 - 9
     CTL            = %x00-1F / %x7F      ; control characters (octets 0 - 31) and DEL (127)
     CR             = %x0D                ; carriage return
     LF             = %x0A                ; line feed
     CRLF           = CR LF               ; MS newline = carriage return line feed
     SP             = %x20                ; space
     HT             = %x09                ; horizontal tab
     WS             = SP  /  HT           ; white space
     EOL            = CR  /  LF  /  CRLF  ; cross-platform end-of-line 
\end{verbatim}

\subsection{Basic Constructs}
\label{apdx:gram_basic}
\marginpar{Is \texttt{FLOAT}\\specified\\correctly?}
\begin{verbatim}
     SIGN           = "+"  /  "-"
     EXPONENT       = ("e"  /  "E"  /  "d"  /  "D")  [SIGN]  1*DIGIT
     NUMBER         = 1*DIGIT  ["."  *DIGIT]  [EXPONENT]
     INF            = ("i"  /  "I")  ("n"  /  "N")  ("f"  /  "F")
     NAN            = ("n"  /  "N")  ("a"  /  "A")  ("n"  /  "N")
     FLOAT          = [SIGN]  (NUMBER  /  INF  / NAN )

     TEXT           = %09 / %x20-FF        ; any OCTET except CTLs, including WS
     COMM           = "#" / ";"
     PROPERWORD     = ALPHA  *(ALPHA  /  DIGIT  /  "_")
     WORD           = *(ALPHA  /  DIGIT  /  "_")
     MATH           = ["ln"] *("-"  /  "+"  /  "*"  /  "$"  /  "/"  /  "("  /  ")"  DIGIT)

     FIELD-END      = "#"  1*"/"  EOL
     HEADER-END     = "#"  2*"-"  EOL
\end{verbatim}
\mcomm{\texttt{MATH} is\\not right}{anchor:math}

\subsection{Version Information}
\label{apdx:gram_version}
\begin{verbatim}
     XDI-VERSION    = "XDI/"  1*DIGIT  ". " 1*DIGIT
     APPLICATIONS   = VCHAR
     VERSION        = "#"  XDI-VERSION  *APPLICATIONS  EOL
\end{verbatim}

\subsection{Header Fields}
\label{apdx:gram_hdr_fields}

\subsubsection{Defined Fields}
\label{apdx:gram_hdr_defined}
\begin{verbatim}
     REFLECTION     = 3DIGIT
     ELEMENT        = ("Si" / "Ge" / "Diamond" / "YB66" / "InSb" / "Beryl" / "Multilayer")
     DATETIME       = 4DIGIT  "-"  2DIGIT  "-"  2DIGIT "T" 2DIGIT  ":"  2DIGIT  ":"  2DIGIT
     HARMONIC-VALUE = 1*2DIGIT

     ABSCISSA           = "#"  "Abscissa"           ": "  MATH
     BEAMLINE           = "#"  "Beamline"           ": "  *WORD
     COLLIMATION        = "#"  "Collimation"        ": "  *WORD
     CRYSTAL            = "#"  "Crystal"            ": "  MATERIAL  REFLECTION
     DSPACING           = "#"  "D_spacing"          ": "  FLOAT
     EDGEENERGY         = "#"  "Edge_energy"        ": "  FLOAT
     ENDTIME            = "#"  "End_time"           ": "  DATETIME
     FOCUSING           = "#"  "Focusing"           ": "  *WORD
     HARMONIC-REJECTION = "#"  "Harmonic_rejection" ": "  *VCHAR
     MUFLUOR            = "#"  "Mu_fluorescence"    ": "  MATH
     MUREF              = "#"  "Mu_reference"       ": "  MATH
     MUTRANS            = "#"  "Mu_transmission"    ": "  MATH
     RINGCURRENT        = "#"  "Ring_current"       ": "  FLOAT
     RINGENERGY         = "#"  "Ring_energy"        ": "  FLOAT
     STARTTIME          = "#"  "Start_time"         ": "  DATETIME
     SOURCE             = "#"  "Source"             ": "  *WORD
     UNDULATOR-HARMONIC = "#"  "Undulator-harmonic" ": "  HARMONIC-VALUE

     DEFINEDFIELDS      = *( (  ABSCISSA     / BEAMLINE    / CRYSTAL
                              / COLLIMATION  / DSPACING    / EDGEENERGY
                              / ENDTIME      / FOCUSING    / HARMONIC_REJECTION
                              / MUFLUOR      / MUREF       / MUTRANS
                              / RINGCURRENT  / RINGENERGY  / STARTTIME 
                              / SOURCE       / UNDULATOR_HARMONIC
                              / EXT_FIELD    / FIELD_LINE
                             ) EOL ) FIELD-END
\end{verbatim}

\subsubsection{Extension Fields}
\label{apdx:gram_hdr_extension}
\begin{verbatim}
     FIELD-LINE     = DEFINEDFIELDS
     EXT-FIELD-NAME = WORD  *("-"  WORD)

     FIELD-LINE     = "#"  FIELD-NAME      ": "  *WORD   EOL
     EXT-FIELD      = "#"  EXT-FIELD-NAME  ": "  *VCHAR  EOL
\end{verbatim}

\subsubsection{All Fields}
\label{apdx:gram_fields}
\begin{verbatim}
     FIELDS         =  (FIELD-LINE / EXT-FIELD)(s) FIELD_END
\end{verbatim}


\subsection{User Comments}
\label{apdx:gram_comments}
\begin{verbatim}
     COMMENT-LINE   = "#"  *VCHAR  EOL
     COMMENTS       = *COMMENT-LINE  HEADER-END
\end{verbatim}

\subsection{Column Labels}
\label{apdx:gram_labels}
\begin{verbatim}
     LABEL          = PROPERWORD
     LABELS         = "#" 1*LABEL  EOL
\end{verbatim}

\subsection{Data Section}
\label{apdx:gram_data}
\begin{verbatim}
     DATA-LINE      = *FLOAT EOL
     DATA           = *DATA-LINE
\end{verbatim}

\end{document}

%%% Local Variables:
%%% mode: latex
%%% reftex-mode: t
%%% TeX-PDF-mode: t
%%% TeX-master: t
%%% End:
