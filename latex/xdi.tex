\documentclass{article}
\usepackage{xdi}
\renewcommand{\Summary}{{\xdi} Specification \xdiversion}

\begin{document}


\begin{center}
  \huge
  The XAS Data Interchange Format\\
  Draft Specification, version \xdiversion\\[3ex]
  \Large
  XDI Working Group\\[2ex]
  \normalsize
  \begin{itemize}
  \item Matthew NEWVILLE (University of Chicago Center for Advanced Radiation Sources, APS)
  \item Bruce RAVEL (NIST) \href{mailto:bravel@bnl.gov}{\texttt{bravel\char64 bnl.gov}}
  \item V. Armando SOL{\'E} (ESRF)
  \item Gerd WELLENREUTHER (DESY)
  \end{itemize}
  \begin{description}
  \item[mailing list:]
    \href{http://millenia.cars.aps.anl.gov/mailman/listinfo/xasformat}{\texttt{http://millenia.cars.aps.anl.gov/mailman/listinfo/xasformat}}
  \item[GitHub organization:] \href{https://github.com/XraySpectroscopy}{\texttt{https://github.com/XraySpectroscopy}}
  \end{description}
\end{center}

\tableofcontents

% \section*{Typography notes}
% \label{typography}

% \begin{itemize}
% \itemsep0em
% \item \highlight[red]{Missing links}
% \item \highlight[blue]{Reference to the dictionary of metadata}
% \item Internal link: Sec.~\ref{sec:scope}
% \item \href{https://github.com/XraySpectroscopy/XAS-Data-Interchange/wiki/Xdispec}%
%   {hyperlink to an address on the Web}
% \item \xditt{verbatim content of files}
% \end{itemize}

\thispagestyle{empty}
\newpage
\linenumbers
\linenumbersep=25pt


~

\section{Introduction}
\label{sec:introduction}


This document describes the XAS Data Interchange Format ({\xdi}),
version \xdiversion, a simple file format for a single X-ray
Absorption Spectroscopy (XAS) measurement.

This document is an effort of an \textit{ad hoc} working group
reporting to the \href{http://www.ixasportal.net/}{International X-ray
  Absorption Society (IXAS)} and the
\href{http://www.iucr.org/resources/commissions/xafs}{XAFS Commission
  of International Union of Crystallography (IUCr-XC)}.  The charge of
this working group is to propose standards for the storage and
dissemination of XAS and related data.


\subsection{Purpose}
\label{sec:purpose}

We are define this format to accomplish the following goals:

\begin{enumerate}
\item Establish a common language for transferring data between XAS
  beamlines, XAS experimenters, data analysis packages, web
  applications, and anything else that needs to process XAS data.
\item Increase the relevance and longevity of experimental data by
  reducing the amount of \textit{data archeology}\footnote{The Farrel
    Lytle ``database''
    (\href{http://ixs.csrri.iit.edu/database/data/index.html}
    {\texttt{http://ixs.csrri.iit.edu/database/data/index.html}}) is a
    particularly trenchant example of data archeology.} future
  interpretations of that data will require.
\item Enhance the user experience by promoting interoperability among
  data acquisition systems, data analysis packages, and other
  applications.
\item Provide a mechanism for extracting and preserving a single
  XAS-like data set from a related experiment (for example, a DAFS or
  inelastic scattering measurement) or from a complex data structure
  (for example, a database or a hierarchical data file used to store a
  multi-spectral data set).
\item Provide a representation of an XAS spectrum suitable for
  deposition with a journal.
\end{enumerate}

In short, we are trying to share data across continents, decades, and
analysis toolkits.

This format is intended to encode a single XAS spectrum in a data file
with metadata.  It is not intended to encode relationships between
many XAS measurements or between an XAS measurement and other parts of
a multi-spectral experiment.

In order to fulfill these goals, {\xdi} files provide a flexible,
consistent representation of information common to all XAS
experiments.  This format is simpler than a format based on XML, HDF,
or a database; it yields self-documenting files; and it is easy for
both humans and computers to read.  Its structure is inspired by that
of Internet electronic mail\footnote{See \textit{RFC822: Standard for
    ARPA Internet Text Messages},
  \href{http://www.w3.org/Protocols/rfc822/}
  {\texttt{http://www.w3.org/Protocols/rfc822/}}}, a plain-text data
format which has proven to be robust, extensible, and enduring.  It
can be read as is by many existing programs for XAS and other data
analysis and by many scientifc plotting programs.

Due to these advantages, and because of our intention to develop free
software tools and libraries that support {\xdi}, we hope that this
file format described in this specification will see wide adoption in
the XAS community.



\subsection{Scope}
\label{sec:scope}

We do not intend this specification to dictate the file formats used
by data acquisition systems during XAS experiments, although this may
be a suitable format for that purpose.  Any attempt to do so would be
unreasonable due to the number of different data acquisition systems
currently deployed at synchrotrons around the world, the variety of
experiments performed at these installations, and the continuing
development of new experimental techniques.  

\textit{This specification addresses the representation of a single
  scan of XAS data after an experiment has been completed.}

A beamline which adopts this specification shall either use this
format as its native file format or shall provide their users with
tools that convert between their native file formats and {\xdi}.  In
short, when that beamline sends a user home with XAS data that is
ready to be analyzed, that XAS data will be stored in this format.  We
intend to encourage this practice by developing tools for reading,
editing, writing, and validating {\xdi} files.  Beamlines may choose
to modify their data acquisition systems to write data using this
format in situations where that would be appropriate.  We plan to
assist in this effort by developing libraries for popular programming
languages which can read, manipulate, and write {\xdi} files.

With their experimental data stored in {\xdi} files, users may choose
data analysis packages which are capable of reading this format.  It
is our hope that, as this specification gains wider adoption, users
will ultimately be freed from the responsibility of understanding file
formats.  With this aim in mind, we shall assist software developers
in supporting {\xdi} files.


\section{The Contents of the XDI File}
\label{sec:contents_ixsif}

{\xdi} files contain two sections, a header with information about one
scan of an XAS experiment followed by the data collected during that
scan. The header section consists of versioning information, a series
of fields that contain information about the scan, an area for users
to store comments about the experiment, and a sequence of labels for
the columns of data. The data section contains these columns, with
each row corresponding to one point of the scan.

The header has been designed to contain arbitrary metadata describing
the contents of the file. This metadata is organized in a way that is
easily readable by both humans and computers. These fields, described
below, contain information about XAS experiments which is useful for
both users and applications. A complete list of defined headers along
with their specifications is found in Sec.~\ref{sec:defnamespaces}.


\section{Definition of the XAS Data Interchange Format}
\label{sec:def_ixsif}

This section of the {\xdi} specification formally describes the
structure of {\xdi} files.

\subsection{Requirements}
\label{sec:def_requirements}

The key words ``\textbf{must}'', ``\textbf{must not}'',
``\textbf{required}'', ``\textbf{should}'', ``\textbf{should not}'',
``\textbf{recommended}'', ``\textbf{may}'', and ``\textbf{optional}''
in this document are to be interpreted as described in RFC
2119.\footnote{\textit{Key words for use in RFCs to Indicate
    Requirement Levels}: \href{http://www.ietf.org/rfc/rfc2119.txt}
  {\texttt{http://www.ietf.org/rfc/rfc2119.txt}}.}

An {\xdi} implementation is \textit{not compliant} if it fails to
satisfy one or more of the \textbf{must} or \textbf{required} level
requirements presented in this specification.

\subsection{Notational Conventions}
\label{sec:def_notation}


Several {\xdi} \textit{tokens} are used throughout the definition of the {\xdi} file.

\begin{itemize}
\item The white space token is a space (ASCII 32) or a tab (ASCII 9)
\item The comment token is a hash (\xditt{\#})
\item The end-of-line token can be carriage return (ASCII 13,
  \xditt{CR}, Mac-style), newline (ASCII 10, \xditt{LF},
  Unix-style), or a sequence of one carriage return + one newline
  (Windows-style)
\item The field-end token consists of three or more foreward slash
  characters (\xditt{/}, ASCII 47)
\item The header-end token consists of three or more dash characters
  (\xditt{-}, ASCII 45)
\end{itemize}


\subsection{Text Encoding}
\label{sec:def_encoding}

The header and data sections of an {\xdi} file are comprised of
structured US-ASCII\footnote{ASCII table:
  \href{http://en.wikipedia.org/wiki/ASCII}
  {\texttt{http://en.wikipedia.org/wiki/ASCII}}} text.  Header field
values that are ``free-form'' or ``text'' \textbf{may} contain UTF-8
encoded Unicode text, although Unicode support in applications that
use XDI files \textbf{should not} be assumed, particularly those
written in languages with weak or non-existent Unicode suport
(e.g. Fortran).  Unicode support in applications that use {\xdi} files
is \textbf{optional}, but \textbf{recommended}.  The US-ASCII coded
character set is defined formally by ANSI X3.4-186\footnote{See
  section 4.1.2 of
  \href{http://tools.ietf.org/html/rfc2046}{Multipurpose Internet Mail
    Extensions (MIME) Part Two: Media Types}}.  The Universal
Character Set (Unicode) is defined by ISO/IEC 10646. \footnote{The
  UTF-8 translation format is defined by
  \href{http://tools.ietf.org/html/rfc3629}{IETF RFC 3629}.}

\subsection{Structure of the Header Section}
\label{sec:def_hdr}


The header section of an {\xdi} file appears at the beginning of the
file and is comprised of structured text.

Header line rules:

\begin{itemize}
\item Every line of the header \textbf{must} begin with a comment
  token and must end with an end-of-line token
\item Header lines may be of any length, but users of {\xdi}
  \textbf{should} remember that XAS software may be implemented in a
  programming language without dynamic memory allocation
  (e.g.\ Fortran) and so should restrict lines to 2048 characters.
\end{itemize}

Header lines are subdivided into four subsections --- versioning
information, header fields, user comments, and column labels --- with
two separators, one of which is always \textbf{required}.  These
subsections \textbf{must} occur in the following sequence:

\begin{enumerate}
\item The \textbf{required} first line of the file is the version
  line, described in Sec.~\ref{sec:def_hdr_version}.
\item This is followed by header lines, which can be defined headers
  or extension headers. These two header types are explained in
  Sec.~\ref{sec:def_hdr_fields}.  Some headers are \textbf{required},
  as explained in Sec.~\ref{sec:requiredelements}.
\item The header lines are separated from the user comments by the
  field-end line.  If the comment section is present, this separator
  line \textbf{must} also be present. If the comment section is
  absent, the header lines \textbf{may} terminate with the
  end-of-header line. The field-end line is defined at the end of this
  section.
\item The \textbf{optional} comment section is for user-supplied,
  free-format text.  Each line begins with a comment token and ends
  with an end-of-line.  See Sec.~\ref{sec:def_hdr_comments}.
\item The comment section ends with the \textbf{required} header-end
  line.  The header-end line is defined at the end of this section.
\item The last line before the data is a line of \textbf{optional}
  column labels which identify the columns of data.  If present, there
  \textbf{must} be as many labels as there are columns. The label line
  begins with a comment character and ends with an end-of-line. See
  Sec.~\ref{sec:def_hdr_labels}.
\end{enumerate}


The field-end and header-end separator lines serve specific, syntactic
purposes in the {\xdi} grammar. For the human reader, the line of
dashes is a visual cue denoting the end of the headers and beginning
of the data. The field-end line serves to separate and distinguish
field lines from freely-formatted user comments, which may resemble a
header fields or other grammatical constructs. Similarly, the
header-end line serves to distinguish column labels from user
comments, which are otherwise grammatically identical elements of the
data file.


\paragraph{Definitions of separator lines}
\begin{description}
\item[Field-end line:] comment token + field-end token + end-of-line
  token\\\qquad\xditt{\# /////////////}
\item[Header-end line:] comment token + header-end token + end-of-line
  token\\\qquad\xditt{\# -------------}
%%\char45 \char45 \char45 \char45 \char45 \char45 \char45 \char45 \char45 \char45 \char45 \char45 \char45 \char45 \char45}
\end{description}




\subsubsection{Version Information}
\label{sec:def_hdr_version}

The first line of the {\xdi} header contains the {\xdi} version to
which the file conforms.  {\xdi} represents versions of the file format
with a \xditt{<version>.<subversion>.<release>} numbering scheme. The
\xditt{<subversion>} number is incremented when changes are made to
the format that do not affect compatibility with previous versions, as
when new defined header fields are defined.  (A parser compliant with
an earlier minor version would treat the newly defined header as an
extension field.  Propagated to an output file as an extension field,
this field would then be interpreted correctly by a more recent
parser.)  The \xditt{<version>} number is incremented when major
changes are made to the format, as when the definition of the contents
of a defined header field is altered.  The \xditt{<release>} is
incremented when the library or its documentation is altered without
altering the specification in any way.  Use of the \xditt{<release>}
number in {\xdi} files is \textbf{optional}.

A series of \textbf{optional} entries denoting further versioning
information, separated by white space, \textbf{may} follow the {\xdi}
version.  There \textbf{may} be any number of extra versioning
strings. These version entries allow programs to annotate the file as
it proceeds through the collection and analysis process.  Such
annotation is \textbf{optional} although version information
\textbf{should} be included in this sequence by software that create
{\xdi} files containing extension fields (see
Sec.~\ref{sec:extension}).  When an application adds versioning
information to this line, it \textbf{should} be appended to the end of
the line.  The order of the optional version entries is undefined but
\textbf{should} be preserved by application reading the file in order
to accurately represent the time sequence in which applications have
manipulated the file.

Note that the XDI version, subversion, and release numbers
\textbf{must} be treated as integers that \textbf{may} contain more
than a single digit. \xditt{XDI/1.12} is a higher (more recent) version than
\xditt{XDI/1.2}.

This specification does not impose a restriction on how applications
identify and version themselves.  However, a single application
\textbf{must} identify and version itself using a single text sequence
without white space. Some acceptable examples follow.  The first
example shows an application which uses the same format as the {\xdi}
version rule, which is the \textbf{recommended} format for application
versioning; the second shows the names of the data acquisition and
data processing programs are specified by name but without the
\textbf{recommended} version numbers; the third shows an example of a
data acquisition program which uses non-standard versioning.


\begin{center}
\begin{Boxedminipage}[h]{0.7\linewidth}
\begin{verbatim}

     # XDI/1.0 Datacollectatron/7.75

     # XDI/1.0 XDAC Athena

     # XDI/1.0 XAS!Collect-3000

\end{verbatim}
\end{Boxedminipage}
\end{center}

The name of the the additional applications \textbf{must} be used for
any extension headers associated with that application (see
Sec.~\ref{sec:extension}).


\subsubsection{Header Fields}
\label{sec:def_hdr_fields}

Immediately following the version line is the header fields
subsection.  These fields are arranged in a manner similar to the
header of an Internet electronic mail message, although {\xdi} fields
\textbf{must not} span multiple lines.  Each field consists of a
case-insensitive name, a separating colon, and an associated value.
The structure of the name is presented in Sec.~\ref{sec:ixsif_fields}.
When multiple occurrences of the same field are present the value of
the last occurrence \textbf{must} be used as the value for the field.

Except in the case of a defined header whose value has a defined
structure, values are assumed to be free-form text, as explained in
Sec.~\ref{sec:def_encoding}.  The defined fields are explained in
Sec.~\ref{sec:defnamespaces}.

When a user comments section is present, the header fields subsection
must end with a field-end line.  When a comments section is absent,
the header fields subsection \textbf{must} end with a field-end line.  See
Sec.~\ref{sec:def_hdr} for the definitions of the separator lines.

\subsubsection{User Comments}
\label{sec:def_hdr_comments}

Following the dividing line at the end of the header fields subsection
is the area of the header that contains user comments.  This area is
reserved for comments supplied by the experimenter and \textbf{must
  not} be used by software as a place to store other information.
Refer to Sec.~\ref{sec:extension} for information about using
extension fields for this purpose.

This section \textbf{may} contain zero lines of commentary or empty
lines containing no text other than the \textbf{required} comment
token.  An empty line \textbf{must} be treated as a zero-length
comment line. This section \textbf{must} end with a header-end
separator line.

When extracting the comment subsection from an {\xdi} file, software
\textbf{may} remove no more than one leading space and any trailing
white space from each comment line but \textbf{must not} further alter
the line's contents, all interior white space \textbf{must} be preserved.

Applications \textbf{must} preserve all user comments, including empty
lines and interior white space, when exporting the {\xdi} data as an
{\xdi} file.



\subsubsection{Column Labels}
\label{sec:def_hdr_labels}

The final line of the {\xdi} header contains the labels for each
column of data in the data section of the file, separated by
white space.  There \textbf{must} be one label present for each column
of data present in the data section.

The number of column labels \textbf{must} equal the number of columns
of data in the data section.

Note that each column label \textbf{must} be a word, white space
\textbf{must} separate the labels, and labels \textbf{must not}
contain white space. For specific column labels which, in natural
language, would consist of two or more words, the use of
\href{http://en.wikipedia.org/wiki/Camel_case}{CamelCase},
underscores, or some other way of substituting for white space is
\textbf{required}.

The column labels in the column label line \textbf{must} match the
values of the headers in the \xditt{Column.} namespace.  See
Sec.~\ref{sec:columnnamespace}.

Several common array labels are defined in the {\DMD} and
\textbf{must} be used when those arrays are present in a file.


\subsection{Data Section}
\label{sec:def_data}

The data section of the file contains white-space-delimited columns of
integers or floating-point numbers.  Lines in the data section
\textbf{must not} begin with comment tokens.  Lines in the data
section \textbf{may} begin with white space.  Leading white space on a
line in the data section \textbf{must} be ignored.

The first (left-most) column of data \textbf{must} contain the
abscissa (energy or angle) array.

Blank lines in this section \textbf{must} be discarded.  The number of
columns \textbf{must} be the same for all lines that contain data.
All columns, including columns containing a measurement of time,
\textbf{must} be represented as inegers or as floating point numbers.

It is \textbf{recommended} that measurements of time be represented as
a numerical offset relative to the value of the
\xditt{Scan.start\_time} header.



\section{XDI Fields}
\label{sec:ixsif_fields}

When present, header fields \textbf{must} comply with the associated
parsing rules.  All fields which fail to do so \textbf{must} be
ignored by an application.

{\xdi} fields use a simple namespace concept as their structure.  The
name of the field \textbf{must} be of two words.  The first word in
the name \textbf{must} start with a letter and \textbf{must not} start
with a number, underscore, or dash.  The second word \textbf{must}
consist of letters, numbers, underscore, or dash.  Letters are ASCII
65 through 90 (\xditt{A-Z}) and ASCII 97-122 (\xditt{a-z}).  Numbers
are ASCII 48-57 (\xditt{0-9}).  Underscore (\xditt{\_}) is ASCII 95
and dash (\xditt{-}) is ASCII 45.

The two words in the name \textbf{must} be separated by the dot
character (\xditt{.}, ASCII 46). The name \textbf{must} end with a
colon (\xditt{:}, ASCII 58), which is the character which delimits
the field name from its value.  The colon \textbf{may} be followed by
white space, then \textbf{must} be followed by the value of the field.
A missing value \textbf{must} be interpreted as an empty string.

Here are some examples which demonstrate both the format of the {\xdi}
field and the \textit{namespace} concept:

\begin{center}
\begin{Boxedminipage}[h]{0.7\linewidth}
\begin{verbatim}

    # Beamline.name: APS 20BM
    # Beamline.source: bend magnet
    # Column.1: energy eV
    # Column.3: i0

\end{verbatim}
\end{Boxedminipage}
\end{center}

The namespaces are used to group related fields.  In the example
above, two namespaces are shown.  The \xditt{Beamline} namespace
conveys characteristics of the beamline at which the data were
measured, while the \xditt{Column} namespace explains how to
interpret the columns in the data section.

There are two kinds of namespaces.  Defined namespaces
(Sec.~\ref{sec:defnamespaces}) are defined in the {\DMD}.  Extension
namespaces (Sec.~\ref{sec:extension}) may be added by application
developers to insert metadata into the data file.


Header fields are case insentitive. As an example, the following lines
\textbf{must} be interpreted identically:

\begin{center}
\begin{Boxedminipage}[h]{0.7\linewidth}
\begin{verbatim}

    # Beamline.name: APS 20BM
    # beamline.name: APS 20BM
    # BEAMLINE.NAME: APS 20BM
    # bEAmlINe.naME: APS 20BM

\end{verbatim}
\end{Boxedminipage}
\end{center}

Capitalization (like the first of these examples) of the namespace is
\textbf{recommended}.


\subsection{Defined namespaces}
\label{sec:defnamespaces}

See the {\DMD} for the current list of defined namespaces and defined
metadata.

Three defined fields are \textbf{required} in a valid {\xdi} file:

\begin{enumerate}
\item \xditt{Element.symbol}: The symbol of the absorber element
\item \xditt{Element.edge}: The measured absorption edge
\item \xditt{Mono.d\_spacing}: The d-spacing of the monochromator
  crystal.  This is only \textbf{required} when the energy axis is
  conveyed as monochromator angle or encoder step count.  When the
  energy axis is conveyed in energy units or pixel count, providing
  the d-spacing is strongly \textbf{recommended} to enable correction
  of the energy axis for a miscalibration due to inaccuracies in the
  translation from angular position of the monochromator to energy.
\end{enumerate}

All other fields are \textbf{optional}, although some are
\textbf{recommended} and constitute good practice, as explained in the
{\DMD}.

A header in a defined namespace \textbf{should not} appear more than
once in a file.  When multiple occurrences of the same field are
present, the value of the last occurrence \textbf{must} be used as the
value for the field.

\subsection{The Column namespace}
\label{sec:columnnamespace}

The Column namespace is the mechanism by which {\xdi} files provide
directions about how to extract useful information from the columns in
the data section of the file.

\begin{enumerate}
\item All fields in this namespace \textbf{must} be of the form
  \xditt{Column.N}, where \xditt{N} represents an integer.  The
  integer is used to identify a particular column in the data
  file.  These integers begin at 1 and count from the left-most column
  in the data section.  The value of a Column field is used to indicate
  the contents of that column.
\item There are several defined column labels.  These are words that
  \textnormal{must} be used to describe a column when that column is
  present in the data file and identified among the header fields.
  The list of defined column labels is given in the
  {\DMD}.
\item The abscissa of the data \textbf{must} be in the first
  (left-most) column and \textbf{must} be identified by the
  \xditt{Column.1} header.
\item Data \textbf{may} be stored using any reasonable units for the
  abscissa, but that choice of units must be identified in the value
  of the \xditt{Column.1} header.  Allowed abscissa choices include
  energy (in units of eV or keV), pixel (appropriate for dispersive
  detection of XAS), or angle (in units of degrees, radians, or motor
  steps).  eV units are \textbf{recommended}.  If units of motor steps
  are chosen, then adequate information \textbf{must} be provided via
  headers in the \xditt{Mono.} namespace to translate the abscissa
  into energy units.
\item The header identifying the abscissa \textbf{must} provide two
  values: the column label for the abscissa and the corresponding
  units. Here is an example:
\begin{verbatim}
  # Column.1: energy eV
\end{verbatim}
  All other headers in the Column namespace \textbf{must} provide one
  value -- the column label -- and \textbf{should} provide units, if
  appropriate.
\end{enumerate}


A list of column labels and their meanings along with unit definitions
for the abscissa are defined in the {\DMD}.  Any such array included
in an {\xdi} file must use those label definitions.  Along with column
labels defining the abscissa and various detectors, labels for
representing EXAFS data in various stages of data processing
($\mu(E)$, normalized $\mu(E)$, $\chi(k)$, the Fourier transform of
$\chi(k)$, or the Fourier filter of $\chi(k)$) are provided.

\subsection{Extension header and extension namespaces}
\label{sec:extension}

Extension fields are fields present in the header of an {\xdi} file
that are not defined in the {\xdi} specification.  Such fields
\textbf{must} be structured by the same syntax as a defined field.
The values of extension fields \textbf{must} be interpreted as
free-form text.  Any field not defined in Sec.~\ref{sec:defnamespaces}
\textbf{must} be considered an extension field.


Data acquisition systems and data analysis packages may embed
additional information in an {\xdi} file by adding extension fields to
the header.  Extension fields created by applications \textbf{should}
begin with a form of the application name used in the version line,
followed by a separator dot and an additional word.  In
App.~\ref{apdx:example} an example of an extension field is
\xditt{GSE.EXTRA} and takes a value of \xditt{config 1}.  Here
\xditt{GSE} denotes the data acquisition software and \xditt{EXTRA}
denotes a parameter relevant to that software.

Extension field namespaces \textbf{must not} collide with the defined
namespaces.

Applications that read {\xdi} files \textbf{may} attempt to parse the
values of extension fields to extract the additional information about
the scan.  They \textbf{should} propagate these fields into output
files they create, and \textbf{must} propagate the associated version
information if they do so.

Multiple occurrences of the same field are discouraged.  When present,
the value of the last occurrence (reading linearly from the beginning
of the file) \textbf{must} be preserved.

\subsection{Required elements}
\label{sec:requiredelements}

The following is a summary of the required elements of an {\xdi} file:

\begin{enumerate}
\item The first line of the file \textbf{must} contain version
  information. See Sec.~\ref{sec:def_hdr_version}.
\item The column containing the abscissa of the data and the units of
  the abscissa \textbf{must} be identified by a header field in the
  \xditt{Column.} namespace.  For example, if the first column of the
  data file contains energy in eV units, the following header field
  \textbf{must} appear in the file:
\begin{verbatim}
# Column.1: energy eV
\end{verbatim}
\item The column containing the abscissa of the data \textbf{must} be
  the first (left-most) column in the data section.
\item The \textbf{Mono.d\_spacing} header field \textbf{must} be
  specified if the abscissa is conveyed as monochromator angle.
\item The \xditt{Element.symbol} and \xditt{Element.edge} headers
  are \textbf{required} in order to definatively identify the XAS
  measurement.
\item If user comments (see Sec.~\ref{sec:def_hdr_comments}) are
  present in the header, the field-end line \textbf{must} be
  present to separate headers from user comments.
\item The header-end separate line \textbf{must} be present.
\item A data section \textbf{must} be present and each line of data
  \textbf{must} contain the same number of data fields and each field
  \textbf{must} be interpretable as an integer or a floating point
  number.
\end{enumerate}

All other content is \textbf{optional}.  When present certain content
\textbf{must} meet further requirements.  See the {\DMD}.

\begin{itemize}
\item Headers containing time stamps, such as \xditt{Scan.start\_time}
  and \xditt{Scan.end\_time} \textbf{must} use the time stamp format
  of ISO
  8601\footnote{\href{http://en.wikipedia.org/wiki/ISO_8601}{ISO 8601}
    defines the exchange of date and time related data.  An example of
    a combined date and time representation is
    \xditt{2007-04-05T14:30}, which means 2:30 in the afternoon on the
    day of April 5${^\mathrm{th}}$ in the year 2007.}.
\item Headers in the \xditt{Column.} namespace \textbf{must} use the
  label columns defined in the {\DMD} as values identifying the column
  types given in that table.  Columns containing other kinds of data
  arrays may be labeled in a free-form manner according to the rules
  for header values.
\end{itemize}


\newpage
\appendix

\section{Example XDI File}
\label{apdx:example}

Here is an example of a file conforming to this specification and
providing substantial metadata.  This was edited by hand from a real
data file measured at beamline 13-ID at the APS in 2001.  The line
beginning \xditt{GSE.EXTRA} is an extension fields denoting
parameters of the data acquisition system in use at the beamline.

\begin{center}
\begin{Boxedminipage}[h]{0.7\linewidth}
\begin{alltt}
\input{../baddata/bad_00.xdi}
\end{alltt}
\end{Boxedminipage}
\end{center}



\end{document}

%%% Local Variables:
%%% mode: latex
%%% reftex-mode: t
%%% TeX-PDF-mode: t
%%% flyspell-mode: t
%%% TeX-master: t
%%% End:
